\documentclass[12pt]{article}
\usepackage{amsmath}
\usepackage{latexsym}
\usepackage{amsfonts}
\usepackage[normalem]{ulem}
\usepackage{soul}
\usepackage{array}
\usepackage{amssymb}
\usepackage{extarrows}
\usepackage{graphicx}
\usepackage[backend=biber,
style=numeric,
sorting=none,
isbn=false,
doi=false,
url=false,
]{biblatex}\addbibresource{bibliography.bib}

\usepackage{subfig}
\usepackage{wrapfig}
\usepackage{txfonts}
\usepackage{wasysym}
\usepackage{enumitem}
\usepackage{adjustbox}
\usepackage{ragged2e}
\usepackage[svgnames,table]{xcolor}
\usepackage{tikz}
\usepackage{longtable}
\usepackage{changepage}
\usepackage{setspace}
\usepackage{hhline}
\usepackage{multicol}
\usepackage{tabto}
\usepackage{float}
\usepackage{multirow}
\usepackage{makecell}
\usepackage{fancyhdr}
\usepackage[toc,page]{appendix}
\usepackage[hidelinks]{hyperref}
\usetikzlibrary{shapes.symbols,shapes.geometric,shadows,arrows.meta}
\tikzset{>={Latex[width=1.5mm,length=2mm]}}
\usepackage{flowchart}\usepackage[paperheight=11.69in,paperwidth=8.27in,left=1.18in,right=0.39in,top=0.79in,bottom=0.79in,headheight=1in]{geometry}
\usepackage[utf8]{inputenc}
\usepackage[russian]{babel}
\usepackage[T1,T2A]{fontenc}
\TabPositions{0.5in,1.0in,1.5in,2.0in,2.5in,3.0in,3.5in,4.0in,4.5in,5.0in,5.5in,6.0in,6.5in,}
\graphicspath{{./img/}}
\urlstyle{same}


 %%%%%%%%%%%%  Set Depths for Sections  %%%%%%%%%%%%%%

% 1) Section
% 1.1) SubSection
% 1.1.1) SubSubSection
% 1.1.1.1) Paragraph
% 1.1.1.1.1) Subparagraph


\setcounter{tocdepth}{5}
\setcounter{secnumdepth}{5}


 %%%%%%%%%%%%  Set Depths for Nested Lists created by \begin{enumerate}  %%%%%%%%%%%%%%


\setlistdepth{9}
\renewlist{enumerate}{enumerate}{9}
		\setlist[enumerate,1]{label=\arabic*)}
		\setlist[enumerate,2]{label=\alph*)}
		\setlist[enumerate,3]{label=(\roman*)}
		\setlist[enumerate,4]{label=(\arabic*)}
		\setlist[enumerate,5]{label=(\Alph*)}
		\setlist[enumerate,6]{label=(\Roman*)}
		\setlist[enumerate,7]{label=\arabic*}
		\setlist[enumerate,8]{label=\alph*}
		\setlist[enumerate,9]{label=\roman*}

\renewlist{itemize}{itemize}{9}
		\setlist[itemize]{label=$\cdot$}
		\setlist[itemize,1]{label=\textbullet}
		\setlist[itemize,2]{label=$\circ$}
		\setlist[itemize,3]{label=$\ast$}
		\setlist[itemize,4]{label=$\dagger$}
		\setlist[itemize,5]{label=$\triangleright$}
		\setlist[itemize,6]{label=$\bigstar$}
		\setlist[itemize,7]{label=$\blacklozenge$}
		\setlist[itemize,8]{label=$\prime$}

\pagenumbering{gobble}
\setlength{\topsep}{0pt}\setlength{\parskip}{9.96pt}
\setlength{\parindent}{0pt}

 %%%%%%%%%%%%  This sets linespacing (verticle gap between Lines) Default=1 %%%%%%%%%%%%%%


\renewcommand{\arraystretch}{1.3}


%%%%%%%%%%%%%%%%%%%% Document code starts here %%%%%%%%%%%%%%%%%%%%



\begin{document}
\begin{Center}
ГУАП
\end{Center}\par

\begin{Center}
КАФЕДРА № 43
\end{Center}\par
\vspace{\baselineskip}
ОТЧЕТ \\
ЗАЩИЩЕН С ОЦЕНКОЙ\par

ПРЕПОДАВАТЕЛЬ\par



%%%%%%%%%%%%%%%%%%%% Table No: 1 starts here %%%%%%%%%%%%%%%%%%%%


\begin{table}[H]
 			\centering
\begin{tabular}{p{2.05in}p{0.0in}p{1.76in}p{-0.01in}p{1.89in}}
%row no:1
\multicolumn{1}{p{2.05in}}{\Centering {ассистент}} & 
\multicolumn{1}{p{0.0in}}{} & 
\multicolumn{1}{p{1.76in}}{} & 
\multicolumn{1}{p{-0.01in}}{} & 
\multicolumn{1}{p{1.89in}}{\Centering {Д. А. Кочин}} \\
\hhline{-~-~-}
%row no:2
\multicolumn{1}{p{2.05in}} {\Centering{\fontsize{10pt}{12.0pt}\selectfont  {должность, уч. степень, звание}}} & 
\multicolumn{1}{p{0.0in}}{} & 
\multicolumn{1}{p{1.76in}} {\Centering{\fontsize{10pt}{12.0pt}\selectfont  {подпись, дата}}} & 
\multicolumn{1}{p{-0.01in}}{} & 
\multicolumn{1}{p{1.89in}} {\Centering{\fontsize{10pt}{12.0pt}\selectfont  {инициалы, фамилия}}} \\
\hhline{~~~~~}

\end{tabular}
 \end{table}


%%%%%%%%%%%%%%%%%%%% Table No: 1 ends here %%%%%%%%%%%%%%%%%%%%


\vspace{\baselineskip}


%%%%%%%%%%%%%%%%%%%% Table No: 2 starts here %%%%%%%%%%%%%%%%%%%%


\begin{table}[H]
 			\centering
\begin{tabular}{p{6.49in}}
%row no:1
\multicolumn{1}{p{6.49in}}{\fontsize{14pt}{16.8pt}\selectfont {\section*{\Centering {ОТЧЕТ О ЛАБОРАТОРНОЙ РАБОТЕ №0}}}}\\
\hhline{~}
%row no:2
\multicolumn{1}{p{6.49in}}{\section*{\Centering {Знакомство с системой контроля версий git и сервисом GitHub}}
} \\
\hhline{~}
%row no:3
\multicolumn{1}{p{6.49in}}{\subsubsection*{\Centering {по курсу: Операционные системы}}
} \\
\hhline{~}
%row no:4
\multicolumn{1}{p{6.49in}}{} \\
\hhline{~}
%row no:5
\multicolumn{1}{p{6.49in}}{} \\
\hhline{~}

\end{tabular}
 \end{table}


%%%%%%%%%%%%%%%%%%%% Table No: 2 ends here %%%%%%%%%%%%%%%%%%%%

РАБОТУ ВЫПОЛНИЛ\par



%%%%%%%%%%%%%%%%%%%% Table No: 3 starts here %%%%%%%%%%%%%%%%%%%%


\begin{table}[H]
 			\centering
\begin{tabular}{p{1.3in}p{1.0in}p{-0.04in}p{1.63in}p{-0.04in}p{1.63in}}
%row no:1
\multicolumn{1}{p{1.3in}}{СТУДЕНТ ГР. №} & 
\multicolumn{1}{p{1.0in}}{\Centering {4232}} & 
\multicolumn{1}{p{-0.04in}}{} & 
\multicolumn{1}{p{1.63in}}{} & 
\multicolumn{1}{p{-0.04in}}{} & 
\multicolumn{1}{p{1.63in}}{\Centering {Д. А. Цыбин}} \\
\hhline{~-~-~-}
%row no:2
\multicolumn{1}{p{1.3in}}{} & 
\multicolumn{1}{p{1.0in}}{} & 
\multicolumn{1}{p{-0.04in}}{} & 
\multicolumn{1}{p{1.63in}} {\Centering{\fontsize{10pt}{12.0pt}\selectfont  {подпись, дата}}} & 
\multicolumn{1}{p{-0.04in}}{} & 
\multicolumn{1}{p{1.63in}} {\Centering{\fontsize{10pt}{12.0pt}\selectfont  {инициалы, фамилия}}} \\
\hhline{~~~~~~}

\end{tabular}
 \end{table}


%%%%%%%%%%%%%%%%%%%% Table No: 3 ends here %%%%%%%%%%%%%%%%%%%%


\vspace{\baselineskip}
\vspace{\baselineskip}
\vspace{\baselineskip}
\vspace{\baselineskip}
\vspace{\baselineskip}
\begin{Center}
Санкт-Петербург \the\year{}
\end{Center}
\pagebreak[4]
\begin{Center}
Задание
\end{Center}
Задание 1:
\begin{itemize}
\item С помощью веб-интерфейса GitHub создайте файл goals.md в своем репозитории.
\item Используя язык разметки markdown, создайте заголовки "Мой опыт" и "Мои цели".
\item Напишите один абзац (1-5 предложений) о своем опыте по разработке программного обеспечения. Добавьте работающую ссылку в написанный текст.
\item Создайте ненумерованный список из 2 или 3 целей, которых планируете достичь в рамках данного курса. 
\item Создайте ненумерованный список из 3-9 задач, над которыми собираетесь работать в рамках данного курса для достижения поставленных целей.
\item Не забудьте закоммитить файл в свой репозиторий на GitHub.
\item Проверьте, что созданный файл отформатирован и отображается при предпросмотре именно так, как планировалось.
\end{itemize}
Задание 2:
\begin{itemize}
\item Ознакомьтесь с инструкцией по работе с git и github из командной строки. 
\item Клонируйте свой репозиторий на локальный компьютер используя команду `git clone ...` из командной строки.
\item Создайте новый файл info.md в своей локальной копии репозитория.
\item Запишите в этот файл свое ФИО (полностью), название используемой операционной системы, текущие дату и время, - каждый элемент с новой строки (всего получится 3 строки).
\item Выполните команды `git add info.md` и `git commit` в командной строке.
\item Введите краткое описание коммита.
\item Выполните команду `git push`.
\end{itemize}
Задание 3:
\begin{itemize}
\item Подготовьте отчет по лабораторной работе, включающий в себя: титульный лист, текст задания, результат выполнения задания (содержимое файлов goals.md и info.md). 
\item Сохраните отчет в файле report.pdf в своем репозитории на GitHub.
\item Для извлечения пользы из этого задания рекомендуется воспользоваться системой компьютерной верстки LaTeX для подготовки отчета. 
\item В случае выполнения отчета в LaTeX следует помимо файла report.pdf загрузить в репозиторий на GitHub исходный *.tex файл.
\end{itemize}
\begin{Center}
Результат выполнения

\includegraphics[scale = 1.0]{1}
Рисунок 1 – текст файла «goals.md» в виде языка разметки markdown

\includegraphics[scale = 1.0]{2}
Рисунок 2 – текст файла «goals.md» в нормальном виде

\includegraphics[scale = 1.0]{3}
Рисунок 3 – текст файла «info.md»
\end{Center}
\end{document}